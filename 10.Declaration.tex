
% -*- program: xelatex -*-
% !TEX root =  main.tex
\thispagestyle{empty}
%\renewcommand{\baselinestretch}{1.6}
%\fontsize{12pt}{13pt}\selectfont
%\setlength{\baselineskip}{20pt plus2pt minus2pt}

 %\markboth{声明及版权}{声明及版权}
 %\addcontentsline{toc}{chapter}{声明}
 \vspace{14pt}
\begin{center}
	{\xiaoerhao {\bf 复旦大学}}\\
	 \vspace{14pt}
	{\xiaoerhao {\bf 学位论文独创性声明}}
\end{center}

\vspace{14pt}

\begin{spacing}{2.0}
%\doublespacing
本人郑重声明:所呈交的学位论文,是本人在导师的指导下,独立进行研究\\
工作所取得的成果。论文中除特别标注的内容外,不包含任何其他个人或机构已\\
经发表或撰写过的研究成果。对本研究做出重要贡献的个人和集体,均已在论文\\
中作了明确的声明并表示了谢意。本声明的法律结果由本人承担。
\end{spacing}
 %本人声明所呈交的学位论文是本人在导师指导下进行的研究工作及取
% 得的研究成果。据我所知,除了文中特别加以标注和致谢的地方外,论文中不包含
% 其他人已经发表或撰写过的研究成果,也不包含为获得~$\underline{\mbox{\kai
%  {复旦大学}}}$~
% 或其他教育机构的学位或证书而使用过的材料。与我一同工作的同志对本研究所
% 做的任何贡献均已在论文中作了明确的说明并表示谢意。
 %\vspace*{8mm}

 {\hfill{
 \begin{tabular}{ll}\\
 作者签名:\underline{\hspace{2.8cm}}& 日期:\underline{\hspace{2cm}}
 \end{tabular}}\par}

 %\vspace*{25mm}
 \vspace{84pt}

\begin{center}
	{\xiaoerhao {\bf 复旦大学}}\\
	 \vspace{14pt}
	{\xiaoerhao {\bf 学位论文使用授权声明}}
\end{center}

\vspace{42pt}

\begin{spacing}{1.5}
本人完全了解复旦大学有关收藏和利用博士、硕士学位论文的规定,即:学\\
校有权收藏、使用并向国家有关部门或机构送交论文的印刷本和电子版本;允许\\
论文被查阅和借阅;学校可以公布论文的全部或部分内容,可以采用影印、缩印\\
或其它复制手段保存论文。涉密学位论文在解密后遵守此规定。
\end{spacing}

% 本学位论文作者完全了解~$\underline{\mbox{\kai
%{复旦大学}}}$~有关保留、
% 使用学位论文的规定,有权保留
% 并向国家有关部门或机构送交论文的复印件和磁盘,允许论文被查阅和借阅。本人授权
% ~$\underline{\mbox{\kai{复旦大学}}}$~
% 可以将学位论文的全部或部分内容编入有关数据库进行检索,可以采用影印、缩印或扫描等
% 复制手段保存、汇编学位论文。
%
% (保密的学位论文在解密后适用本授权书)

 \vspace{14pt}
 {\hfill
 \begin{tabular}{lll}\\
  作者签名:\underline{\hspace{2.8cm}} & 导师签名:\underline{\hspace{2.8cm}} & 日期:\underline{\hspace{2cm}}
 \end{tabular}\par}

 %\vspace*{11mm}


 \begin{center}
 \begin{tabular}{ll}\\
% 学位论文作者毕业后去向:& \\
% 工作单位:复旦大学华山论剑研究所 &
% 电话:+(86)139-1744-9685\\
% 通讯地址:上海市邯郸路220号复旦大学计算机科学技术学院~9999~信箱 & 邮编:310027\\
% E-mail:\href{mailto:cchangyou@gmail.com}{cchangyou@gmail.com} or \href{mailto:072021151@fudan.edu.cn}{072021151@fudan.edu.cn}&

 \end{tabular}
 \end{center}
